% 5. In het voor- of nawoord kunnen zij die op enigerlei wijze betrokken zijn geweest bij de totstandkoming van het proefschrift, op terughoudende wijze worden bedankt zoals gangbaar in de ‘acknowledgements’ in de internationale weten- schappelijke literatuur. Het voor- of nawoord is gesteld in een bij de aard van de promotieplechtigheid passende stijl en telt niet meer dan 400 woorden.


%% kan gepositioneerd worden als Acknowledgements of als Preface

\section*{Preface}
%\addcontentsline{toc}{chapter}{Preface}

After six years of working as an external PhD researcher, two days a week, I finished the dissertation in front of you. Doing research next to a demanding job has by no means been an easy feat, not to mention the number of academic activities I engaged in beyond those strictly necessary for the dissertation. 
%I marvel at how I managed all things listed in the curriculum vitae included here. 
Looking back, I do not recommend others to employ a similar strategy and %in 
%using good project management skills as grounds for
increase the workload so, however fulfilling the additions may be. %--- one of the many lessons I learned over these years. 
%Even so, each of the activities surrounding my research at Leiden University has enriched my life in one way or another. %and reminiscing on them brings a smile to my face. 
Even so, I am thankful for the opportunities each activity presented me and, first and foremost, the strong professional and personal connections that resulted from them (to which this preface testifies). %that resulted from my research at Leiden University.

% dank aan Thijs (en Rolf en Piek)
I would like to express my gratitude to my supervisors Rolf Bremmer, Thijs Porck, and Piek Vossen for their insights and suggestions. Thijs, I hold dear your support, especially. We have co-organized two workshops and co-edited a special issue journal. %You did not eschew the technical details of my research, embracing this Digital Humanities endeavour fully. 
Your words and encouragements helped me progress and ensured I celebrated key milestones. %, a ritual of great value. 
% UoG
I would also like to thank the team behind \textit{A Thesaurus of Old English} wholeheartedly; Jane Roberts, Marc Alexander, and Fraser Dallachy in particular. You were kind enough to allow me to use this treasure trove in my research and inspired me to explore its potential applications --- an exploration for which I am greatly indebted to participants in the research programme Exploring Early Medieval English Eloquence. Your case studies, suggestions, and feedback have been of utmost importance in developing the resources at the heart of this thesis. %A special word of thanks goes to those participating in the EEMEE programme who performed case studies that they presented in workshops and published articles: Amos van Baalen, Rafael Cruz González, Kees Dekker, Katrien Depuydt, Javier E. Díaz-Vera, Jesse de Does, Rachel Fletcher, Fahad Khan, Francisco Javier Minaya Gómez, Monica Monachini, and Rita van de Poel.
% OntoLex & LUCDH & LUCAS(?) & DH/SN?

Further communities enriched my academic life: the W3C OntoLex Community Group, closely collaborating on standardizing Linguistic Linked Data; the Leiden University Centre for the Arts in Society (LUCAS), at which I performed my research and which granted me a fund to finish this dissertation; the Leiden University Centre for Digital Humanities (LUCDH), which offered advice and supported me through small grants; and the Digital Humanities Student Network %at Leiden University 
(DH/SN), in which I found myself amidst peers with a shared passion. 
Thank you for welcoming me as part of your respective communities.

% werk (baas en collega's, running/bouldering buddies)
% Charlotte en Eduardo?
% vriendin, familie/ouders
Additionally, I would like to thank my colleagues at Semmtech, my bouldering buddies, my friends and family for your support and, most of all, your company. 
Lastly, I would like to thank my girlfriend-turned-fiancée-turned-wife, Fenja Schulz. Our journey together has been one of kindness, understanding, and growth. May we continue celebrating what we have and what we achieve, both together and as individuals.

\footnotebl{
\leftskip 0em
~\textbf{Acknowledgements}\\
The dissertation was supported by the LUCAS Extra Resources Open Call-II Grant (5.000 EUR) for the research project Exploring Medieval English Eloquence and by the Afrondingssubsidie buitenpromovendi (18.000 EUR), awarded by the Leiden University Centre for the Arts in Society (2020; 2022). 
The early development of the web application Evoke was supported by the LUCDH Small Grant (3.500 EUR), awarded by the Leiden University Centre for Digital Humanities (2018).
}

\begin{comment}
The editors would like to thank the organisers of the 21st International Conference on English Historical Linguistics 21, which hosted the concluding workshop, as well as all participants for their helpful comments on the presentations. In addition, the editors would like to thank the anonymous peer reviewers whose feedback was essential for the contributions in this special issue. The project `Exploring Medieval English Eloquence' was supported by the LUCAS Extra Resources Open Call-II Grant 2020, awarded by the Leiden University Centre for the Arts in Society, and the LUCDH Small Grant 2018, awarded by the Leiden University Centre for Digital Humanities. These grants enabled the organisation of the workshops as well as the hiring of student assistant Lucas Gahrmann, whose assistance was fundamental throughout the project. Additional thanks are due to Rob Zandvliet and Arend Quak for their patience and support in bringing the special issue to press. 
\end{comment}

\clearpage