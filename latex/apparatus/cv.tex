% 6. Aan het einde van het proefschrift dient een bondig curriculum vitae van de auteur te zijn opgenomen Dit curriculum vitae bevat:
% a. het geboortejaar,
% b. de geboorteplaats,
% c. de data, binnen welke voorbereidend wetenschappelijk of daarmee vergelijkbaar onderwijs werd genoten en de instelling waaraan dit werd genoten,
% d. eventueel kwalificaties bij het behalen der getuigschriften,
% e. gegevens omtrent de beroepsuitoefening na het voltooien van de wetenschappelijke opleiding, en
% f. eventueel aan welk instituut het promotieonderzoek werd uitgevoerd.

\section*{Curriculum vitae}
\addcontentsline{toc}{chapter}{Curriculum vitae}

Sander Stolk was born in Beverwijk on 5 March 1985. He obtained his VWO-Atheneum certificate from Kennemer College, Beverwijk, in 2003. In 2009, he received his M.Sc. degree in Computer Science at VU University, Amsterdam, where he specialized in Internet and Web technology. A subsequent study at Leiden University granted him a M.A. degree in Literary Studies, English Literature and Culture, in 2013. %Philology was a key focus area in both the bachelor and master there. 
His M.A. thesis was awarded for being one of the most novel within the Humanities faculty. 
%During his studies in Leiden, he obtained his teaching qualification for secondary education. %valid in the Netherlands 
%through an educational minor at ICLON, Leiden. 
%All degrees mentioned were awarded \textit{cum laude}.

Since 2014, Sander Stolk has worked at Semmtech, where he has employed linguistic principles to sustainably and effectively share information through ontologies and related Web technologies (i.e., Linked Data, Semantic Web) --- as Software Developer (2014-15), Semantic Architect (2015-19), and, currently, Head of Innovation (2020-present). 
%His experience at Semmtech includes solution design on the R\&D project Virtual Construction for Roads for Rijkswaterstaat and Trafikverket (2015-17, awarded the label `excellent' by European Committee); developing the European Road OTL and acting as Workpackage Leader for the Conference of European Directors of Roads in their BIM Call 2015 (2016-18, awarded the buildingSmart 2018 award); developing national agreements and standards for the use of Linked Data for the built environment (NTA 8035 in 2019-20, NEN 2660-2 in 2021-22); specifying requirements on information exchange at TenneT for EU-303 projects (2020); and setting out the strategy and guidelines for asset information management at Waternet (2022-present). Additionally, he co-supervised four Master's theses (one in 2016, three in 2019).

Sander Stolk started his PhD research as an external student at Leiden University in 2016. %, devoting two days per week to his research. 
He actively contributed to standardization efforts surrounding Linguistic Linked Data as a member of the Ontology-Lexica Community Group of W3C (2017-21); organized workshops on the subject of thesauri and applying Linguistic Linked Data to \textit{A Thesaurus of Old English} (2019, 2020, 2021); co-edited a special issue of the journal \textit{Amsterdamer Beiträge zur älteren Germanistik} 81.3-4 %, titled \textit{Exploring Early Medieval English Eloquence: A Digital Humanities Approach with A Thesaurus of Old English and Evoke} 
(2021); organized a session and round table discussion at the Leeds International Medieval Congress %called `Teaching the Medieval in the Digital Age' 
(2021); and co-founded and chaired the Digital Humanities Student Network %associated with the Leiden University Centre for Digital Humanities 
(2017-22). %Besides an article titled `Marking Boundaries in Beowulf' (2017) and a book review of \textit{Mapping English Metaphor through Time} (2017), 

The majority of his publications and public speaking have either been on the efforts for his PhD research or, through his work at Semmtech, on the utilization of Linked Data for sharing information on the built environment.
