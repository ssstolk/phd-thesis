% see dissertation Porck, p. iii.

% 1. Het proefschrift dient een titelblad te bevatten met vermelding van de bij de burgerlijke stand geregistreerde voorna(a)m(en) en familienaam, alsmede het geboortejaar. Daarnaast bevat het proefschrift een inhoudsopgave en de nodige registers.

\begin{titlepage}

    \vspace{4cm}
    \begin{center}
    
    \textsf{
    %
    {\fontsize{33}{43}\selectfont{
      Digital Thesauri\\
      as Semantic Treasure Troves\\[0.5em]
    }}
    \Large{\MakeUppercase{
      A Linguistic Linked Data Approach to\\[0.3em]
      \emph{A Thesaurus of Old English}\\[3cm]
    }}
    %
    }
    
    
    \normalsize
    \begin{doublespace}
    PROEFSCHRIFT\bigskip
        
    ter verkrijging van\\ 
    de graad van doctor aan de Universiteit Leiden,\\
    op gezag van rector magnificus prof. dr. ir. H. Bijl,\\ 
    volgens besluit van het college voor promoties\\
    te verdedigen op woensdag 31 mei 2023\\
    klokke 10.00 uur\\[1cm] 
     
    door\\[1cm]
         
    \large{Sander Sebastiaan Stolk}\\[0.5cm]
	
    \normalsize geboren te Beverwijk \\
    in 1985
    
    \end{doublespace}
  
	

    \end{center}
    
    

\end{titlepage}