\noindent{\color{white}\rule{\textwidth}{0.01em}}
\checkoddpage\ifoddpage\newpage\mbox{}\else\fi

\thispagestyle{empty}

%\begin{center}
%\setlength\fboxsep{1cm}
%\noindent\colorbox{lightgray}{
%\parbox[c][1.0in-2cm]{\textwidth-2cm}{\RaggedLeft\Huge \textbf{Chapter 9}}}
%\end{center}

\begin{center}
\setlength\fboxsep{1cm}
\noindent\colorbox{evokelightblue}{
\parbox[c][1.0in-2cm]{\textwidth-2cm}{\RaggedLeft\Huge \textbf{\textcolor{white}{Chapter 9}}}}
\end{center}

\noindent{\color{lightgray}\rule{\textwidth}{0.4em}}
\\[2em]
\noindent\textbf{Information:}\\
The article that starts on the next page was published in 2021, available open access under the CC BY license. The only change to that paper, here, is the inclusion of two numbers for the benefit of readers of the dissertation: the overall page number and the chapter number (presented in the margin in a grey box and a white box, respectively). When citing, please refer to the original publication and its page numbering.
\\[1em]
\noindent\textbf{Publication:}\\
%\begin{quotation}\noindent
    Rita van de Poel and Sander Stolk, `A Case of Kinship: Onomasiological Explorations of \textsc{kinship} in Old Frisian and Old English'. \textit{Amsterdamer Beiträge zur älteren Germanistik} 81.3-4 (2021), pp. 457-492. doi: \href{https://doi.org/10.1163/18756719-12340239}{\url{10.1163/18756719-12340239}}.
%\end{quotation}
\noindent
\\[1em]
\noindent{\color{lightgray}\rule{\textwidth}{0.4em}}
