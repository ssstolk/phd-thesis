\noindent{\color{white}\rule{\textwidth}{0.01em}}
\checkoddpage\ifoddpage\newpage\mbox{}\else\fi

\thispagestyle{empty}

%\begin{center}
%\setlength\fboxsep{1cm}
%\noindent\colorbox{lightgray}{
%\parbox[c][1.0in-2cm]{\textwidth-2cm}{\RaggedLeft\Huge \textbf{Chapter 4}}}
%\end{center}

\begin{center}
\setlength\fboxsep{1cm}
\noindent\colorbox{evokelightblue}{
\parbox[c][1.0in-2cm]{\textwidth-2cm}{\RaggedLeft\Huge \textbf{\textcolor{white}{Chapter 4}}}}
\end{center}

\noindent{\color{lightgray}\rule{\textwidth}{0.4em}}
\\[2em]
\noindent\textbf{Information:}\\
The paper that starts on the next page was published in 2017, available open access. The only change to that paper, here, is the inclusion of two numbers for the benefit of readers of the dissertation: the overall page number and the chapter number (presented in the margin in a grey box and a white box, respectively). When citing, please refer to the original publication and its page numbering.
\\[1em]
\noindent\textbf{Publication:}\\
%\begin{quotation}\noindent
    Sander Stolk, `OntoLex and Onomasiological Ordering: Supporting Topical Thesauri', Proceedings of the LDK2017 Workshops, NUI Galway, 18 June 2017, pp. 60–67. \url{http://ceur-ws.org/Vol-1899/OntoLex_2017_paper_3.pdf}.
%\end{quotation}
\noindent
\\[1em]
\noindent{\color{lightgray}\rule{\textwidth}{0.4em}}
