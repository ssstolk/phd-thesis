% The propositions must be such that they can be defended with scientific arguments.


\documentclass[twoside,openright,11pt]{book}

\usepackage[paperheight=230mm,paperwidth=160mm,inner=20mm,outer=20mm,bottom=20mm,top=20mm,heightrounded]{geometry}

\usepackage[dvipsnames]{xcolor}
\usepackage[unicode=true]{hyperref}
\usepackage{url}
\usepackage[hang]{footmisc} 
\usepackage{changepage}
% avoid all hyphenation
\tolerance=1
\emergencystretch=\maxdimen
\hyphenpenalty=10000
\hbadness=10000
\raggedbottom

\pagestyle{empty}

\begin{document}



\begin{center}

{PROPOSITIONS}

\bigskip
%
%accompanying the dissertation by S. Stolk
%
%\bigskip


    \textsf{
    %
    {\Large\bfseries
      Digital Thesauri as Semantic Treasure Troves\\[0.3em]
    }
    {
      A Linguistic Linked Data Approach to %\\[0.3em]
      \emph{A Thesaurus of Old English}\\
    }
    %
    }

\bigskip

\end{center}

\small 

\begin{adjustwidth}{-1.25em}{-0.25em}
\begin{enumerate}

% -------------------------------------------------------
% at least four propositions relating to the subject of the dissertation
% -------------------------------------------------------

\item Previous investigations into Old English language and culture have by no means exhausted existing historical language thesauri. 
%For use in academia, there is yet ample opportunity to improve the Web-based dissemination of such lexicographic resources to facilitate novel research. %historical language thesauri of Scots and English.

\item Linguistic Linked Data as a digital form is ideally suited for sharing historical language thesauri (and other topical thesauri) for research purposes.%: It is superior to a paper format in that it allows expanding and digital querying of the knowledge within and, in comparison to other digital forms, it is effective for linking additional data to lexicographic content even in cases where licensing prohibits access to raw thesaurus data other than through the presentations offered by a published edition.

\item The abilities to reuse and elaborate on thesaurus content and to subsequently perform onomasiological analyses on the newly available, combined knowledge constitute the most notable novel pieces of functionality desired for research that are currently lacking in the majority of the published historical language thesauri of Scots and of English.

\item %Onomasiological comparisons of the semantic domain of \textsc{kinship} 
Old Frisian and Old English are closely related historical languages, yet notable differences between them exist in the semantic domain of \textsc{kinship}.
% Old Frisian includes more fine-grained word senses than in Old English % (including ones to denote different degrees of kinship) and has a relatively higher degree of lexicalization of the concepts of ancestry and descent than the surviving Old English lexis. Additionally, Old Frisian lacks words to express a number of concepts for \textsc{kinship} that \emph{are} found in Old English: adoption, foundling, twins, and triplets.%\footnote{R. van de Poel and S. Stolk, `A Case of Kinship: Onomasiological Explorations of \textsc{kinship} in Old Frisian and Old English', \textit{Amsterdamer Beiträge zur älteren Germanistik} 81.3-4 (2021), 457–92 (pp. 478-9). doi: \href{https://doi.org/10.1163/18756719-12340239}{\url{10.1163/18756719-12340239}}.}


% -------------------------------------------------------
% at least four scientific propositions relating to the field of the subject of the dissertation
% -------------------------------------------------------

\item The larger the onomasiological structure of a thesaurus, the more difficult it will be to use this structure for navigating% (due to inherent editorial decisions)
, but the more useful it will be for analysis of semantic domains.

% DONE: RB:Korter en uitdagender.
%\item Additional software applications sporting user-friendly user interfaces are needed in order to have scholars fully embrace Linguistic Linked Data. For purposes of data collection, alignment, and analysis, software currently available for this digital form tends to be aimed at data scientists rather than scholars in Humanities.
\item For scholars in the Humanities to fully embrace working with Linguistic Linked Data, further software development is required that is aimed at them rather than at data scientists. 

% TODO: RB:Van constatering naar stelling. bijv. Wat zegt het van de mentaliteit/kijk op het leven dat men nu een goede gezondheid bij een afscheid wenst?
%\item Greetings have changed over the course of the English history. Whereas Old English employed phrases to wish the addressed party good health upon greeting, Modern English offers such biddings of good health at departures instead.\footnote{S. Stolk, `Welcoming the \textit{Thesaurus of Old English Statistics}: The \textit{Thesaurus of Old English} and the Vocabulary of Greetings', M.A. dissertation, Leiden University (2013).}

\item In the Old English epic \textit{Beowulf}, Æschere's severed head serves as a boundary marker.%\footnote{M. H. Porck and S. Stolk, `Marking Boundaries in \textit{Beowulf}: Æschere’s Head, Grendel’s Arm and the Dragon’s Corpse', \textit{Amsterdamer Beiträge zur älteren Germanistik} 77.3-4 (2017), 521-40 (pp. 523-33). doi: \href{https://doi.org/10.1163/18756719-12340090}{\url{10.1163/18756719-12340090}}.}

% DONE: RB:Daar zullen weinig mensen het mee oneens zijn, dus niet zo uitdagend. Uitdagender zou bijvoorbeeld zijn als je stelt dat wetenschappers verplicht moeten worden hun dataverzamelingen vrijtoegankelijk bij hun publicaties te voegen.
%\item Open science -- i.e., science in which articles are published in open access journals, created datasets are shared in a manner that other researchers can freely access and engage with the material, and source code of software is made available under an open license -- leads to more transparent, more accessible, and overall better research.
\item Researchers should be required to make their datasets and source code publicly available, under an open license, when used in a publication.

% DONE: RB:Veel te lang en te bescheiden
\item %Although no single definition exists of Digital Humanities (DH), the intersection of digital technologies and the disciplines of the Humanities, and various approaches to this field co-exist, %
Digital humanists can contribute more effectively to academia by supporting scholars in the Humanities %in navigating and improving the means at their disposal for utilizing electronic resources and incorporating computational lines of inquiry 
than by demarcating Digital Humanities as its own unified academic field.
% Ted Underwood, in syllabus `History and Theory of Digital Humanities' (2020): ``The phrase \textit{digital humanities} was coined twenty years ago. But its meaning is still open to debate. For some observers, this is the name of a unified academic field, with its own conferences, journals, and professional organizations. Others see “DH” as a general name for any application of computers in the humanities. Both views could be right: it’s complicated.''
% Wikipedia, article on Digital Humanities: ``The definition of the digital humanities is being continually formulated by scholars and practitioners. Since the field is constantly growing and changing, specific definitions can quickly become outdated or unnecessarily limit future potential.[4] The second volume of Debates in the Digital Humanities (2016) acknowledges the difficulty in defining the field: "Along with the digital archives, quantitative analyses, and tool-building projects that once characterized the field, DH now encompasses a wide range of methods and practices: visualizations of large image sets, 3D modeling of historical artifacts, 'born digital' dissertations, hashtag activism and the analysis thereof, alternate reality games, mobile makerspaces, and more. In what has been called 'big tent' DH, it can at times be difficult to determine with any specificity what, precisely, digital humanities work entails."''

%\item ... On linking to corpus. (own article)
%\item ... LD for asset information management? (own article)

% -------------------------------------------------------
% at most four propositions on one or more subjects of the candidate’s choice
% -------------------------------------------------------

\item \textit{``The man is not wholly evil: he has a Thesaurus in his cabin''} \\
\scriptsize --- the play \textit{Peter Pan or the Boy who Would Not Grow Up} (1928) on Captain Hook\\ 
\small%\footnote{The description is found in Act IV of the play published as J. M. Barrie's \textit{Peter Pan or the Boy who Would Not Grow Up} (London, 1928).}
Based on his actions, Captain Hook must be considered wholly evil and, as such, any accounts of him owning a thesaurus must be mistaken or attributed to the imaginations of an unreliable narrator. 

\item \textit{``Bûter, brea en griene tsiis, wa’t dat net sizze kin, is gjin oprjochte Fries.''}\\The coinage of this shibboleth by and the practices attributed to Grutte Pier, the sixteenth-century Frisian warrior, position him as a proscriptive linguist of a rather extreme variety.
% Wikipedia: ``Pier wordt ook gezien als bedenker van het sjibbolet "Bûter, brea en griene tsiis, wa't dat net sizze kin, is gjin oprjochte Fries" wat zoveel betekent als: "Boter, roggebrood en groene kaas, wie dat niet zeggen kan is geen oprechte (ware) Fries"''
% for extremeness of punishments, consider the sheer length of his enormous sword as evidence

%\item Beowulf is simply not a good swimmer.
%\item Bede's account of a sparrow flying through one's hall is, if anything, indicative of either poor isolation in early medieval England or an open door policy.

\end{enumerate}
\end{adjustwidth}


% -------------------------------------------------------
% apart from the name and title of the supervisor, the following text is dictated by the promotieregelement.
% -------------------------------------------------------

\bigskip
\begin{center}
\footnotesize
These propositions are regarded as opposable and defendable, and have been approved as such by the promotor prof.\ dr.\ R.\ H.\ Bremmer Jr.
\end{center}

%\clearpage

\end{document}
