\noindent{\color{white}\rule{\textwidth}{0.01em}}
\checkoddpage\ifoddpage\newpage\mbox{}\else\fi

\thispagestyle{empty}

%\begin{center}
%\setlength\fboxsep{1cm}
%\noindent\colorbox{lightgray}{
%\parbox[c][1.0in-2cm]{\textwidth-2cm}{\RaggedLeft\Huge \textbf{Chapter 6}}}
%\end{center}

\begin{center}
\setlength\fboxsep{1cm}
\noindent\colorbox{evokelightblue}{
\parbox[c][1.0in-2cm]{\textwidth-2cm}{\RaggedLeft\Huge \textbf{\textcolor{white}{Chapter 6}}}}
\end{center}

\noindent{\color{lightgray}\rule{\textwidth}{0.4em}}
\\[2em]
\noindent\textbf{Information:}\\
The paper that starts on the next page was published in 2019, available open access under the CC BY-SA license. The only change to that paper, here, is the inclusion of two numbers for the benefit of readers of the dissertation: the overall page number and the chapter number (presented in the margin in a grey box and a white box, respectively). When citing, please refer to the original publication and its page numbering.
\\[1em]
\noindent\textbf{Publication:}\\
%\begin{quotation}\noindent
    Sander Stolk, `\emph{A Thesaurus of Old English} as Linguistic Linked Data: Using OntoLex, SKOS and \emph{lemon-tree} for Bringing Topical Thesauri to the Semantic Web', Proceedings of the eLex 2019 conference, Sintra, 1-3 October 2019, pp. 223–247. \url{http://ceur-ws.org/Vol-1899/OntoLex_2017_paper_3.pdf}.
%\end{quotation}
\noindent
\\[1em]
\noindent{\color{lightgray}\rule{\textwidth}{0.4em}}
